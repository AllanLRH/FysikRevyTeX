\documentclass[11pt]{article}
\usepackage[danish]{babel}
\usepackage[latin1]{inputenc}
%\usepackage{a4}                          % A4 papirst�rrelse
%\usepackage{fullpage}
\usepackage{vmargin}
\setpapersize{A4}
%argumenter:{hleftmargini}{htopmargini}{hrightmargini}{hbottommargini}{headheight}{headsep}{footheight}{footskip}
\setmarginsrb{20mm}{15mm}{20mm}{20mm}{12pt}{11mm}{0pt}{11mm}
\usepackage[T1]{fontenc}                 % ������
\usepackage{color}                       % farver
\usepackage{verbatim}
\usepackage{amsmath}                     % Matematiske kommandoer
\usepackage{amssymb}                     % Matematiske symboler
%\usepackage{longtable}
\usepackage{array,booktabs,longtable}
\usepackage{multicol}
%\usepackage{isolatin1}
%\usepackage{t1enc}
%\usepackage{fancyhdr}
%\usepackage{bbold}
\usepackage{float}

% Floats_______________________________________
\floatstyle{plain}
\newfloat{song}{H}{eks}
\floatname{song}{Sang}

\floatsep=1pt

%-----------------------------------------------------%
%                   KOMMANDOER                        %
%-----------------------------------------------------%
\newcommand{\sang}[2]{\begin{song}[h!]
\caption[#1]{\LARGE\sffamily #1}\end{song}\begin{center} Melodi: #2 \end{center}}

\newcommand{\rp}[1]{\textsf{#1}: }
\newcommand{\rg}[1]{\emph{(#1)}}
\newcommand{\rgsang}[2]{\multicolumn{2}{#1 c}{\emph{(#2)}}}
\newcommand{\overskrift}[1]{\begin{center}\LARGE\sffamily #1 \end{center}}

\newcommand{\FysikRevy}{$\textrm{\textsf{FysikRevy}}^{\textrm{\textsf{\tiny{TM}}}}$}
\newcommand{\Kairsten}{K$\frac{a}{i}$rsten}
\newcommand{\Simon}{$\psi$-mon}
\newcommand{\p}{$\Psi$}

%\newcommand{\rol}{\subsection*{\ref{#1}}}

\setlongtables

\addtolength{\topmargin}{-10pt}
\addtolength{\textheight}{20pt}

\newfont{\cmfnt}{ecssdc10.pk at 50pt}
\newfont{\cmfntt}{ecssdc10.pk at 30pt}
\newfont{\cmfnttt}{ecssdc10.pk at 20pt}
%\newfont{\frfont}{ecssdc10.pk at 80pt}
%\newfont{\frfontii}{ecssdc10.pk at 30pt}
\newfont{\frfont}{eclq8.pk at 80pt}
\newfont{\frfontii}{eclq8.pk at 30pt}

\parindent=0pt
\parskip=5pt
%\marginparwidth=10pt
%\textwidth=450pt
%\hoffset=-0.25in

%%%%%%%%%%%%%%%%%%%%%%%%%%%%%%%%%%%%%%%%%%%%%%%%%%%%%
%%%			Start dokument		%%%%%
%%%%%%%%%%%%%%%%%%%%%%%%%%%%%%%%%%%%%%%%%%%%%%%%%%%%%
\begin{document}
\thispagestyle{empty}

\begin{flushright}
	{\tiny Per Hedeg�rd $\neq$ Baconost}
\end{flushright}
\hrule
\begin{center}
%{\cmfnt Fysikrevyen 2009 \\ Tekster\\}
{\frfont $\textrm{\frfont{FysikRevy}}^{\textrm{\frfontii{{TM}}}}$}
\\ --- {\frfontii tekster} ---
\vspace{2cm}

{\cmfntt{Ver. 0011000100110000}}\\ %\Huge{\textbf{$|5+i\sqrt{11}|$}}}}\\
\vspace{5mm}
{\cmfnttt{\today}}\\
\vspace{1cm}
%de n�ste versions numre er:
%
\end{center}


\begin{center}
%%%%%%%%%%%%%%%%%%%%%%%%%%%%%%%%%
% INDLEDENDE MORALPR�IKEN
%%%%%%%%%%%%%%%%%%%%%%%%%%%%%%%%%

\Large
Du holder nu i h�nden de guddommelige \TeX ster til \FysikRevy{} 2011. Dette indeb�rer (som tidligere �r) f�lgende:
\begin{itemize}
\item Du (\TeX)h�fter for disse \TeX ster med dit liv.
\item Hvis du mister dem vil du blive chapset i osten med et v�dt hestebr�d!
\item Hvis du viser dem til ikke-indviede er dit liv �delagt, og du kan lige s� 
godt lade dig indskrive p� datalogi og tage menneske-datamaskine interaktion.
\item Hvis du gerne vil undg�, at andre tager dit \TeX h�fte som gidsel,
skal du skrive dit navn p� forsiden. S� er du sikker p� at andre ikke vil kunne
finde p� at stj�le DIT \TeX h�fte. 
\item Hvis du ikke skriver navn p� dit \TeX h�fte, m� du f�rst g� hjem, n�r du har f�et eduroam til at fungere.
\end{itemize}
\end{center}
\vfill
%%%%%%%%%%%%%%%%%%%%%%%%%%%%%%%%%%%%%%%%%
% FORSIDEBILLEDE
%%%%%%%%%%%%%%%%%%%%%%%%%%%%%%%%%%%%%%%%%


%\begin{center}
%\epsfig{file=60meterhest.eps,width=5cm}\\
%\end{center}

\vfill
�rets \TeX h�fte er sat i \LaTeXe.
\begin{center}
	{\tiny ``Hvis ikke jeg havde haft den ugentlige dosis Rage Against the Machine p� Caf�en?, var jeg nok aldrig blevet prodekan''}
\end{center}
\end{document}
